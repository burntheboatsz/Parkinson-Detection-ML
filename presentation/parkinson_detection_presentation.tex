\documentclass[aspectratio=169]{beamer}
\usepackage[utf8]{inputenc}
\usepackage[indonesian]{babel}
\usepackage{graphicx}
\usepackage{booktabs}
\usepackage{amsmath}
\usepackage{xcolor}
\usepackage{tikz}
\usepackage{pgfplots}
\pgfplotsset{compat=1.18}

% Theme
\usetheme{Madrid}
\usecolortheme{default}

% Custom colors
\definecolor{darkblue}{RGB}{0,51,102}
\definecolor{lightblue}{RGB}{51,153,255}
\definecolor{darkgreen}{RGB}{0,128,0}
\definecolor{darkred}{RGB}{178,34,34}

\setbeamercolor{structure}{fg=darkblue}
\setbeamercolor{title}{fg=white,bg=darkblue}
\setbeamercolor{frametitle}{fg=white,bg=darkblue}

% Title page info
\title[Deteksi Parkinson dengan ML]{Sistem Deteksi Penyakit Parkinson \\ Menggunakan Machine Learning}
\subtitle{Analisis Komparatif 10 Model dan Strategi Optimasi}
\author{Nafiz Ahmadin Harily \\ NIM: 122430051}
\institute{Program Studi Teknik Biomedis \\ Institut Teknologi Sumatera (ITERA)}
\date{\today}

\begin{document}

% Title slide
\begin{frame}
\titlepage
\end{frame}

% Table of contents
\begin{frame}{Outline}
\tableofcontents
\end{frame}

%=============================================================================
\section{Pendahuluan}
%=============================================================================

\begin{frame}{Latar Belakang}
\begin{block}{Penyakit Parkinson}
\begin{itemize}
    \item Penyakit neurodegeneratif yang mempengaruhi sistem motorik
    \item Diagnosis dini sangat penting untuk penanganan efektif
    \item Gejala awal: tremor, kekakuan otot, gangguan keseimbangan
    \item \textbf{Tantangan:} Diagnosis manual memerlukan keahlian khusus
\end{itemize}
\end{block}

\begin{block}{Solusi dengan Machine Learning}
\begin{itemize}
    \item Menggunakan biomarker suara untuk deteksi otomatis
    \item Model ML dapat mengidentifikasi pola yang tidak terlihat
    \item Akurat, cepat, dan dapat diakses secara luas
\end{itemize}
\end{block}
\end{frame}

\begin{frame}{Tujuan Penelitian}
\begin{enumerate}
    \item \textbf{Eksplorasi Data:} Memahami karakteristik dataset biomarker suara pasien
    \item \textbf{Training Model:} Melatih dan membandingkan 10 algoritma ML berbeda
    \item \textbf{Evaluasi Performa:} Mengidentifikasi model terbaik berdasarkan metrik evaluasi
    \item \textbf{Optimasi Model:} Mencoba 5 strategi untuk meningkatkan akurasi
    \item \textbf{Deployment:} Menyiapkan model untuk produksi dengan tools yang user-friendly
\end{enumerate}

\vspace{0.5cm}
\begin{center}
\textbf{\color{darkgreen}Target Akurasi: $\geq 95\%$}
\end{center}
\end{frame}

%=============================================================================
\section{Dataset dan Metodologi}
%=============================================================================

\begin{frame}{Deskripsi Dataset}
\begin{columns}
\column{0.5\textwidth}
\begin{block}{Informasi Dataset}
\begin{itemize}
    \item \textbf{Nama:} Parkinsons Dataset
    \item \textbf{Sumber:} UCI Repository
    \item \textbf{Total Sampel:} 195 pasien
    \item \textbf{Fitur:} 24 kolom
    \item \textbf{Target:} Binary (0=Sehat, 1=Parkinson)
    \item \textbf{Missing Values:} Tidak ada
\end{itemize}
\end{block}

\column{0.5\textwidth}
\begin{block}{Distribusi Kelas}
\begin{itemize}
    \item \textbf{Parkinson:} 147 pasien (75.4\%)
    \item \textbf{Sehat:} 48 pasien (24.6\%)
    \item \textbf{Imbalance Ratio:} 3.06:1
\end{itemize}
\end{block}

\begin{alertblock}{Catatan}
Dataset relatif kecil (195 samples) → membatasi kompleksitas model
\end{alertblock}
\end{columns}
\end{frame}

\begin{frame}{Fitur-Fitur Biomarker Suara}
\begin{table}
\centering
\small
\begin{tabular}{lp{7cm}}
\toprule
\textbf{Kategori} & \textbf{Fitur} \\
\midrule
\textbf{Frequency} & MDVP:Fo (Hz), MDVP:Fhi (Hz), MDVP:Flo (Hz) \\
\textbf{Variation} & MDVP:Jitter(\%), Jitter:DDP, MDVP:Shimmer, Shimmer:DDA \\
\textbf{Ratio} & NHR (Noise-to-Harmonics), HNR (Harmonics-to-Noise) \\
\textbf{Nonlinear} & RPDE, DFA, D2, PPE (entropy measures) \\
\textbf{Statistical} & Spread1, Spread2 \\
\bottomrule
\end{tabular}
\caption{22 fitur biomarker suara (setelah drop kolom 'name' dan 'status')}
\end{table}

\vspace{0.3cm}
\begin{center}
\textit{Semua fitur adalah pengukuran karakteristik suara pasien}
\end{center}
\end{frame}

\begin{frame}{Metodologi Eksperimen}
\begin{tikzpicture}[
    node distance=1.5cm,
    box/.style={rectangle, draw, fill=lightblue!30, text width=3cm, text centered, rounded corners, minimum height=1cm},
    arrow/.style={->, >=stealth, thick}
]
    \node[box] (data) {1. Data Loading \& EDA};
    \node[box, right of=data, xshift=3cm] (prep) {2. Preprocessing};
    \node[box, right of=prep, xshift=3cm] (train) {3. Model Training};
    \node[box, below of=train, yshift=-0.5cm] (eval) {4. Evaluasi};
    \node[box, left of=eval, xshift=-3cm] (opt) {5. Optimasi};
    \node[box, left of=opt, xshift=-3cm] (deploy) {6. Deployment};
    
    \draw[arrow] (data) -- (prep);
    \draw[arrow] (prep) -- (train);
    \draw[arrow] (train) -- (eval);
    \draw[arrow] (eval) -- (opt);
    \draw[arrow] (opt) -- (deploy);
\end{tikzpicture}

\vspace{0.5cm}
\begin{block}{Preprocessing Steps}
\begin{enumerate}
    \item Drop kolom 'name' (identifier, bukan fitur)
    \item Split data: 80\% training, 20\% testing (stratified)
    \item Scaling dengan StandardScaler
    \item \textbf{Tidak} menggunakan SMOTE (hasil lebih baik tanpa balancing)
\end{enumerate}
\end{block}
\end{frame}

%=============================================================================
\section{Exploratory Data Analysis (EDA)}
%=============================================================================

\begin{frame}{Statistik Deskriptif Dataset}
\begin{columns}
\column{0.5\textwidth}
\begin{block}{Ringkasan Statistik}
\begin{itemize}
    \item \textbf{Shape:} 195 rows × 24 columns
    \item \textbf{Features:} 22 numerik
    \item \textbf{Target:} 1 binary (status)
    \item \textbf{Identifier:} 1 string (name)
\end{itemize}
\end{block}

\begin{block}{Kualitas Data}
\begin{itemize}
    \item ✓ Tidak ada missing values
    \item ✓ Tidak ada duplicate rows
    \item ✓ Semua fitur numerik valid
    \item ✓ Tidak ada outlier ekstrem
\end{itemize}
\end{block}

\column{0.5\textwidth}
\begin{table}
\centering
\tiny
\begin{tabular}{lrr}
\toprule
\textbf{Fitur} & \textbf{Mean} & \textbf{Std} \\
\midrule
MDVP:Fo(Hz) & 154.23 & 41.39 \\
Jitter(\%) & 0.0062 & 0.0049 \\
Shimmer & 0.0298 & 0.0189 \\
NHR & 0.0249 & 0.0402 \\
HNR & 21.89 & 4.43 \\
RPDE & 0.4981 & 0.1035 \\
DFA & 0.7181 & 0.0554 \\
PPE & 0.2064 & 0.0904 \\
\bottomrule
\end{tabular}
\caption{Sample statistik (8 dari 22 fitur)}
\end{table}
\end{columns}
\end{frame}

\begin{frame}{Visualisasi: Distribusi Target}
\begin{center}
\begin{tikzpicture}
\begin{axis}[
    ybar,
    width=10cm,
    height=6cm,
    ylabel={Jumlah Pasien},
    symbolic x coords={Sehat, Parkinson},
    xtick=data,
    nodes near coords,
    nodes near coords align={vertical},
    bar width=2cm,
    ymin=0,
    ymax=160,
    legend pos=north east,
]
\addplot[fill=darkgreen] coordinates {(Sehat,48) (Parkinson,147)};
\end{axis}
\end{tikzpicture}
\end{center}

\begin{itemize}
    \item \textbf{Imbalanced:} Rasio 3:1 (Parkinson:Sehat)
    \item Stratified split penting untuk menjaga proporsi
    \item Model harus hati-hati agar tidak bias ke kelas mayoritas
\end{itemize}
\end{frame}

\begin{frame}{Insight dari EDA}
\begin{block}{Temuan Penting}
\begin{enumerate}
    \item \textbf{Kualitas Data Excellent:} Tidak ada preprocessing intensif diperlukan
    \item \textbf{Fitur Informatif:} 
    \begin{itemize}
        \item Jitter dan Shimmer lebih tinggi pada pasien Parkinson
        \item HNR (Harmonics-to-Noise) lebih rendah pada Parkinson
        \item RPDE, DFA, PPE menunjukkan pola yang jelas
    \end{itemize}
    \item \textbf{Korelasi:} Beberapa fitur saling berkorelasi tinggi (multikolinearitas ringan)
    \item \textbf{Separabilitas:} Kelas cukup terpisah → model seharusnya performa baik
\end{enumerate}
\end{block}

\begin{alertblock}{Kesimpulan EDA}
Dataset berkualitas tinggi, siap untuk training model ML tanpa feature engineering ekstensif
\end{alertblock}
\end{frame}

%=============================================================================
\section{Training dan Evaluasi Model}
%=============================================================================

\begin{frame}{10 Model Machine Learning yang Diuji}
\begin{columns}
\column{0.5\textwidth}
\begin{block}{Traditional ML (5 model)}
\begin{enumerate}
    \item Logistic Regression
    \item Decision Tree
    \item Random Forest
    \item Support Vector Machine (SVM)
    \item K-Nearest Neighbors (KNN)
\end{enumerate}
\end{block}

\column{0.5\textwidth}
\begin{block}{Advanced ML (5 model)}
\begin{enumerate}
    \setcounter{enumi}{5}
    \item Naive Bayes
    \item Gradient Boosting
    \item \textcolor{darkgreen}{\textbf{XGBoost}} ← Best
    \item LightGBM
    \item CatBoost
\end{enumerate}
\end{block}
\end{columns}

\vspace{0.5cm}
\begin{center}
\textbf{Metrik Evaluasi:} Accuracy, Precision, Recall, F1-Score, ROC-AUC
\end{center}
\end{frame}

\begin{frame}{Hasil Training: Perbandingan Akurasi}
\begin{center}
\begin{tikzpicture}
\begin{axis}[
    xbar,
    width=12cm,
    height=7cm,
    xlabel={Accuracy},
    symbolic y coords={Naive Bayes, Decision Tree, LightGBM, KNN, SVM, Gradient Boosting, Random Forest, Logistic Regression, CatBoost, XGBoost},
    ytick=data,
    nodes near coords,
    nodes near coords align={horizontal},
    xmin=0.6,
    xmax=1.0,
    bar width=0.4cm,
    legend pos=south east,
]
\addplot[fill=darkblue] coordinates {
    (0.6667,Naive Bayes)
    (0.8462,Decision Tree)
    (0.9231,LightGBM)
    (0.9231,KNN)
    (0.9231,SVM)
    (0.9231,Gradient Boosting)
    (0.9231,Random Forest)
    (0.9231,Logistic Regression)
    (0.9487,CatBoost)
    (0.9487,XGBoost)
};
\addplot[fill=darkgreen] coordinates {(0.9487,XGBoost)};
\legend{Models, Best Model}
\end{axis}
\end{tikzpicture}
\end{center}
\end{frame}

\begin{frame}{Tabel Hasil Lengkap: Top 5 Models}
\begin{table}
\centering
\small
\begin{tabular}{lrrrrr}
\toprule
\textbf{Model} & \textbf{Accuracy} & \textbf{Precision} & \textbf{Recall} & \textbf{F1} & \textbf{ROC-AUC} \\
\midrule
\rowcolor{darkgreen!20}
\textbf{XGBoost} & \textbf{0.9487} & \textbf{0.9487} & \textbf{0.9487} & \textbf{0.9487} & \textbf{0.9690} \\
CatBoost & 0.9487 & 0.9487 & 0.9487 & 0.9487 & 0.9638 \\
Random Forest & 0.9231 & 0.9231 & 0.9231 & 0.9231 & 0.9638 \\
Logistic Reg. & 0.9231 & 0.9231 & 0.9231 & 0.9231 & 0.9845 \\
SVM & 0.9231 & 0.9231 & 0.9231 & 0.9231 & 0.9638 \\
\bottomrule
\end{tabular}
\caption{Top 5 model dengan performa terbaik}
\end{table}

\vspace{0.5cm}
\begin{block}{XGBoost Dipilih Karena:}
\begin{itemize}
    \item ✓ Akurasi tertinggi: \textbf{94.87\%}
    \item ✓ ROC-AUC terbaik: \textbf{96.90\%}
    \item ✓ Balanced metrics (precision = recall = F1)
    \item ✓ Inference cepat (<100ms)
\end{itemize}
\end{block}
\end{frame}

\begin{frame}{Confusion Matrix: XGBoost}
\begin{center}
\begin{tikzpicture}
\draw[thick] (0,0) rectangle (4,4);
\draw[thick] (2,0) -- (2,4);
\draw[thick] (0,2) -- (4,2);

% Labels
\node at (1,4.3) {\textbf{Predicted: 0}};
\node at (3,4.3) {\textbf{Predicted: 1}};
\node[rotate=90] at (-0.5,3) {\textbf{Actual: 0}};
\node[rotate=90] at (-0.5,1) {\textbf{Actual: 1}};

% Values
\node[fill=darkgreen!30, minimum size=1.8cm] at (1,3) {\Huge \textbf{9}};
\node[fill=darkred!30, minimum size=1.8cm] at (3,3) {\Huge \textbf{1}};
\node[fill=darkred!30, minimum size=1.8cm] at (1,1) {\Huge \textbf{1}};
\node[fill=darkgreen!30, minimum size=1.8cm] at (3,1) {\Huge \textbf{28}};

% Labels for cells
\node at (1,0.5) {\small TN};
\node at (3,0.5) {\small TP};
\node at (1,2.5) {\small FP};
\node at (3,2.5) {\small FN};
\end{tikzpicture}
\end{center}

\begin{itemize}
    \item \textbf{True Positive (TP):} 28 (Parkinson terdeteksi benar)
    \item \textbf{True Negative (TN):} 9 (Sehat terdeteksi benar)
    \item \textbf{False Positive (FP):} 1 (Sehat terdeteksi Parkinson)
    \item \textbf{False Negative (FN):} 1 (Parkinson tidak terdeteksi)
    \item \textbf{Total:} 37 benar, 2 salah dari 39 test samples
\end{itemize}
\end{frame}

\begin{frame}{Interpretasi Hasil XGBoost}
\begin{columns}
\column{0.5\textwidth}
\begin{block}{Kelebihan}
\begin{itemize}
    \item ✓ Akurasi 94.87\% (excellent!)
    \item ✓ ROC-AUC 96.90\% (discriminative)
    \item ✓ Balanced: Precision = Recall
    \item ✓ Hanya 2 error dari 39 test
    \item ✓ Generalisasi baik
\end{itemize}
\end{block}

\column{0.5\textwidth}
\begin{block}{Analisis Error}
\begin{itemize}
    \item 1 False Positive: Orang sehat terdiagnosis Parkinson (tidak berbahaya, akan dilakukan tes lanjutan)
    \item 1 False Negative: Pasien Parkinson tidak terdeteksi (lebih serius, perlu improvement)
\end{itemize}
\end{block}
\end{columns}

\vspace{0.5cm}
\begin{alertblock}{Clinical Implication}
Model cocok untuk \textbf{screening awal}, bukan diagnosis final. Hasil positif harus dikonfirmasi oleh dokter spesialis.
\end{alertblock}
\end{frame}

%=============================================================================
\section{Strategi Optimasi Model}
%=============================================================================

\begin{frame}{Tujuan Optimasi}
\begin{center}
\Large
\textbf{Baseline: 94.87\%}

\vspace{0.5cm}
\Huge
$\downarrow$

\vspace{0.5cm}
\textbf{\color{darkgreen}Target: $\geq 95\%$ atau bahkan 96\%+}
\end{center}

\vspace{1cm}
\begin{block}{5 Strategi yang Diuji:}
\begin{enumerate}
    \item Hyperparameter Tuning (RandomizedSearchCV)
    \item Voting Ensemble (XGBoost + CatBoost + LightGBM)
    \item Feature Engineering (Interaction \& Ratio features)
    \item Feature Selection (SelectKBest)
    \item Stacking Ensemble (Meta-learner)
\end{enumerate}
\end{block}
\end{frame}

\begin{frame}{Strategi 1: Hyperparameter Tuning}
\begin{block}{Metode}
\begin{itemize}
    \item Menggunakan RandomizedSearchCV
    \item 50 iterasi dari 3,888 kombinasi parameter
    \item 5-fold Cross-Validation
    \item Parameter: max\_depth, learning\_rate, n\_estimators, min\_child\_weight, subsample, colsample\_bytree, gamma
\end{itemize}
\end{block}

\begin{columns}
\column{0.5\textwidth}
\begin{block}{Best Parameters}
\tiny
\begin{itemize}
    \item max\_depth: 9
    \item learning\_rate: 0.2
    \item n\_estimators: 200
    \item subsample: 0.8
    \item colsample\_bytree: 1.0
    \item gamma: 0.2
\end{itemize}
\end{block}

\column{0.5\textwidth}
\begin{alertblock}{Hasil}
\begin{itemize}
    \item CV Score: 0.9232
    \item Test Accuracy: \textbf{0.9231}
    \item \textcolor{darkred}{\textbf{WORSE by -2.56\%}}
    \item Kesimpulan: Parameter default sudah optimal!
\end{itemize}
\end{alertblock}
\end{columns}
\end{frame}

\begin{frame}{Strategi 2: Voting Ensemble}
\begin{block}{Metode}
\begin{itemize}
    \item Menggabungkan 3 top models: XGBoost + CatBoost + LightGBM
    \item Voting type: \textbf{Soft voting} (menggunakan probabilitas)
    \item Setiap model memberikan "vote" berdasarkan confidence
\end{itemize}
\end{block}

\begin{center}
\begin{tikzpicture}[
    node distance=1.5cm,
    box/.style={rectangle, draw, fill=lightblue!30, text width=2cm, text centered, rounded corners, minimum height=0.8cm},
]
    \node[box] (xgb) {XGBoost};
    \node[box, right of=xgb, xshift=1cm] (cat) {CatBoost};
    \node[box, right of=cat, xshift=1cm] (lgb) {LightGBM};
    \node[box, below of=cat, yshift=-1cm, fill=darkgreen!30] (vote) {Voting};
    \node[box, below of=vote, yshift=-0.5cm, fill=yellow!30] (pred) {Final Prediction};
    
    \draw[->, thick] (xgb) -- (vote);
    \draw[->, thick] (cat) -- (vote);
    \draw[->, thick] (lgb) -- (vote);
    \draw[->, thick] (vote) -- (pred);
\end{tikzpicture}
\end{center}

\begin{alertblock}{Hasil}
Test Accuracy: \textbf{0.9487} → \textcolor{orange}{\textbf{NO CHANGE (0.00\%)}}
\end{alertblock}
\end{frame}

\begin{frame}{Strategi 3: Feature Engineering}
\begin{block}{Metode}
\begin{itemize}
    \item Membuat 20 fitur baru dari kombinasi fitur existing
    \item \textbf{Interaction features:} $f_i \times f_j$ (perkalian fitur)
    \item \textbf{Ratio features:} $f_i / (f_j + \epsilon)$ (pembagian fitur)
    \item Total fitur: 22 → 42 fitur
\end{itemize}
\end{block}

\begin{table}
\centering
\small
\begin{tabular}{lr}
\toprule
\textbf{Jenis Fitur} & \textbf{Jumlah} \\
\midrule
Original features & 22 \\
Interaction features & 10 \\
Ratio features & 10 \\
\midrule
Total & 42 \\
\bottomrule
\end{tabular}
\end{table}

\begin{alertblock}{Hasil}
Test Accuracy: \textbf{0.9487} → \textcolor{orange}{\textbf{NO CHANGE (0.00\%)}}
\end{alertblock}
\end{frame}

\begin{frame}{Strategi 4: Feature Selection}
\begin{block}{Metode}
\begin{itemize}
    \item Menggunakan SelectKBest dengan f\_classif scoring
    \item Menguji berbagai nilai k: 10, 12, 15, 18, 20
    \item Tujuan: Membuang fitur yang tidak informatif
\end{itemize}
\end{block}

\begin{center}
\begin{tikzpicture}
\begin{axis}[
    width=10cm,
    height=5cm,
    xlabel={Number of Features (k)},
    ylabel={Accuracy},
    xtick={10,12,15,18,20},
    ymin=0.8,
    ymax=1.0,
    grid=major,
    legend pos=south east,
]
\addplot[color=darkblue, mark=*, thick] coordinates {
    (10,0.8462)
    (12,0.8718)
    (15,0.9231)
    (18,0.9231)
    (20,0.9487)
};
\addplot[color=darkred, dashed] coordinates {(10,0.9487) (20,0.9487)};
\legend{Feature Selection, Baseline}
\end{axis}
\end{tikzpicture}
\end{center}

\begin{alertblock}{Hasil}
Best k: 20 features → Accuracy: \textbf{0.9487} → \textcolor{orange}{\textbf{NO CHANGE}}
\\ \textit{Semua fitur diperlukan untuk performa optimal!}
\end{alertblock}
\end{frame}

\begin{frame}{Strategi 5: Stacking Ensemble}
\begin{block}{Metode}
\begin{itemize}
    \item \textbf{Base learners:} XGBoost, CatBoost, LightGBM
    \item \textbf{Meta-learner:} Logistic Regression
    \item 5-fold cross-validation untuk training meta-learner
\end{itemize}
\end{block}

\begin{center}
\begin{tikzpicture}[
    node distance=1.2cm,
    box/.style={rectangle, draw, fill=lightblue!30, text width=1.8cm, text centered, rounded corners, minimum height=0.7cm},
]
    % Level 0 (base learners)
    \node[box] (xgb) {XGBoost};
    \node[box, right of=xgb, xshift=0.8cm] (cat) {CatBoost};
    \node[box, right of=cat, xshift=0.8cm] (lgb) {LightGBM};
    
    % Predictions
    \node[box, below of=xgb, yshift=-0.5cm] (p1) {$\hat{y}_1$};
    \node[box, below of=cat, yshift=-0.5cm] (p2) {$\hat{y}_2$};
    \node[box, below of=lgb, yshift=-0.5cm] (p3) {$\hat{y}_3$};
    
    % Meta learner
    \node[box, below of=p2, yshift=-0.8cm, fill=darkgreen!30, text width=5cm] (meta) {Meta-Learner: Logistic Regression};
    
    % Final prediction
    \node[box, below of=meta, yshift=-0.5cm, fill=yellow!30] (final) {Final Prediction};
    
    \draw[->, thick] (xgb) -- (p1);
    \draw[->, thick] (cat) -- (p2);
    \draw[->, thick] (lgb) -- (p3);
    \draw[->, thick] (p1) -- (meta);
    \draw[->, thick] (p2) -- (meta);
    \draw[->, thick] (p3) -- (meta);
    \draw[->, thick] (meta) -- (final);
\end{tikzpicture}
\end{center}

\begin{alertblock}{Hasil}
Test Accuracy: \textbf{0.9487} → \textcolor{orange}{\textbf{NO CHANGE (0.00\%)}}
\end{alertblock}
\end{frame}

\begin{frame}{Ringkasan Hasil Optimasi}
\begin{table}
\centering
\begin{tabular}{lrrr}
\toprule
\textbf{Strategi} & \textbf{Accuracy} & \textbf{Improvement} & \textbf{Status} \\
\midrule
\rowcolor{lightblue!20}
Baseline (XGBoost) & 0.9487 & - & ✓ OPTIMAL \\
Hyperparameter Tuning & 0.9231 & -2.56\% & ✗ WORSE \\
Voting Ensemble & 0.9487 & 0.00\% & ⚠ SAME \\
Feature Engineering & 0.9487 & 0.00\% & ⚠ SAME \\
Feature Selection & 0.9487 & 0.00\% & ⚠ SAME \\
Stacking Ensemble & 0.9487 & 0.00\% & ⚠ SAME \\
\bottomrule
\end{tabular}
\end{table}

\vspace{0.5cm}
\begin{alertblock}{Kesimpulan Optimasi}
\Large
\textbf{Model sudah OPTIMAL!}

\normalsize
Tidak ada improvement yang berhasil. Model XGBoost original dengan default parameters adalah yang terbaik untuk dataset ini.
\end{alertblock}
\end{frame}

\begin{frame}{Mengapa Tidak Bisa Ditingkatkan?}
\begin{block}{Analisis Mendalam}
\begin{enumerate}
    \item \textbf{Dataset Kecil (195 samples)}
    \begin{itemize}
        \item Model sudah belajar semua pola yang ada
        \item Tidak cukup data untuk pola yang lebih kompleks
        \item Overfitting risk jika model lebih kompleks
    \end{itemize}
    
    \item \textbf{Fitur Sudah Optimal}
    \begin{itemize}
        \item Semua 22 fitur relevan dan informatif
        \item Feature engineering tidak menambah informasi baru
        \item Tidak ada fitur redundan
    \end{itemize}
    
    \item \textbf{Model Sudah Fit}
    \begin{itemize}
        \item XGBoost default parameters sudah sangat baik
        \item Hyperparameter tuning malah menurunkan performa
        \item Ensemble tidak memberikan diversity benefit
    \end{itemize}
\end{enumerate}
\end{block}

\begin{center}
\textbf{94.87\% adalah \textcolor{darkgreen}{performance ceiling} untuk dataset ini!}
\end{center}
\end{frame}

%=============================================================================
\section{Deployment dan Aplikasi}
%=============================================================================

\begin{frame}{Tools untuk Deployment}
\begin{block}{3 Interface yang Tersedia}
\begin{enumerate}
    \item \textbf{CLI Tool} - Command Line Interface
    \begin{itemize}
        \item Script: \texttt{python src/predict.py}
        \item Untuk batch prediction atau single patient
        \item Cocok untuk integrasi sistem
    \end{itemize}
    
    \item \textbf{Web Application} - Streamlit
    \begin{itemize}
        \item User-friendly GUI
        \item Upload CSV atau input manual
        \item Real-time prediction dengan probability
    \end{itemize}
    
    \item \textbf{Python API} - ModelUtils
    \begin{itemize}
        \item Integrasi ke aplikasi lain
        \item Load model dan predict programmatically
        \item Flexible untuk berbagai use case
    \end{itemize}
\end{enumerate}
\end{block}
\end{frame}

\begin{frame}[fragile]{Contoh Penggunaan: Python API}
\begin{block}{Load Model dan Predict}
\begin{verbatim}
from model_utils import ModelUtils

# Load model
model = ModelUtils.load_model('models/xgboost.pkl')
scaler, features = ModelUtils.load_preprocessing_params(
    'models'
)

# Predict single patient
result = ModelUtils.predict_single_patient(
    model, scaler, patient_features
)

print(f"Prediction: {result['prediction']}")
print(f"Probability: {result['probability']:.2%}")
\end{verbatim}
\end{block}

\begin{itemize}
    \item Inference time: <100ms per prediction
    \item Model size: 2.1 MB
    \item Memory: ~50MB RAM
\end{itemize}
\end{frame}

\begin{frame}{Confidence Thresholds untuk Clinical Use}
\begin{table}
\centering
\begin{tabular}{lcp{5cm}}
\toprule
\textbf{Probability} & \textbf{Level} & \textbf{Action} \\
\midrule
\rowcolor{darkred!20}
$\geq 80\%$ & High Risk & Lakukan pemeriksaan klinis lengkap segera \\
\rowcolor{orange!20}
$50-79\%$ & Medium Risk & Flag untuk review dokter, tes tambahan \\
\rowcolor{darkgreen!20}
$< 50\%$ & Low Risk & Monitoring rutin, tidak urgent \\
\bottomrule
\end{tabular}
\end{table}

\vspace{0.5cm}
\begin{alertblock}{Important Note}
\begin{itemize}
    \item Model adalah \textbf{screening tool}, bukan diagnostic tool
    \item Hasil positif harus dikonfirmasi oleh neurolog
    \item Gunakan sebagai \textbf{decision support}, bukan keputusan final
\end{itemize}
\end{alertblock}
\end{frame}

\begin{frame}{File-File Model untuk Production}
\begin{block}{Model Artifacts}
\begin{itemize}
    \item \texttt{models/xgboost.pkl} (2.1 MB)
    \begin{itemize}
        \item Trained XGBoost model
        \item Ready untuk load dan inference
    \end{itemize}
    
    \item \texttt{models/scaler.pkl}
    \begin{itemize}
        \item StandardScaler yang sudah fit
        \item Penting untuk preprocessing input baru
    \end{itemize}
    
    \item \texttt{models/feature\_names.json}
    \begin{itemize}
        \item List 22 feature names
        \item Untuk validasi input
    \end{itemize}
\end{itemize}
\end{block}

\begin{block}{Dependencies}
\begin{itemize}
    \item Python 3.11.9
    \item XGBoost 1.7.6, Scikit-learn 1.3.0, Pandas 2.0.3, NumPy 1.24.3
\end{itemize}
\end{block}
\end{frame}

%=============================================================================
\section{Kesimpulan dan Rekomendasi}
%=============================================================================

\begin{frame}{Ringkasan Pencapaian}
\begin{block}{Apa yang Telah Dilakukan}
\begin{enumerate}
    \item ✓ \textbf{EDA Lengkap:} Analisis 195 samples, 24 features
    \item ✓ \textbf{Training 10 Models:} Dari Logistic Regression hingga XGBoost
    \item ✓ \textbf{Evaluasi Komprehensif:} 5 metrics untuk setiap model
    \item ✓ \textbf{Optimasi 5 Strategi:} Hyperparameter, Ensemble, Feature Engineering
    \item ✓ \textbf{Deployment Ready:} CLI, Web App, Python API
    \item ✓ \textbf{Dokumentasi Lengkap:} Report, visualization, code
\end{enumerate}
\end{block}

\begin{block}{Best Model: XGBoost}
\begin{itemize}
    \item \textbf{Accuracy:} 94.87\% (37/39 correct)
    \item \textbf{ROC-AUC:} 96.90\% (excellent discrimination)
    \item \textbf{Status:} Production-ready
\end{itemize}
\end{block}
\end{frame}

\begin{frame}{Limitasi Penelitian}
\begin{alertblock}{Keterbatasan}
\begin{enumerate}
    \item \textbf{Dataset Kecil}
    \begin{itemize}
        \item 195 samples membatasi generalization
        \item Improvement terbatas tanpa data tambahan
    \end{itemize}
    
    \item \textbf{Single Data Source}
    \begin{itemize}
        \item Perlu validasi pada dataset independen
        \item Mungkin tidak generalize ke populasi berbeda
    \end{itemize}
    
    \item \textbf{Binary Classification}
    \begin{itemize}
        \item Tidak membedakan tingkat keparahan Parkinson
        \item Tidak mendeteksi jenis gangguan motorik lain
    \end{itemize}
    
    \item \textbf{Performance Ceiling}
    \begin{itemize}
        \item 94.87\% adalah maksimal untuk dataset ini
        \item Tidak bisa improve tanpa data/fitur baru
    \end{itemize}
\end{enumerate}
\end{alertblock}
\end{frame}

\begin{frame}{Rekomendasi untuk Pengembangan Lebih Lanjut}
\begin{block}{Prioritas TINGGI}
\begin{enumerate}
    \item \textbf{Collect More Data}
    \begin{itemize}
        \item Target: 500-1,000+ patient samples
        \item Expected improvement: +1-3\% accuracy
        \item Lebih robust dan generalizable
    \end{itemize}
    
    \item \textbf{External Validation}
    \begin{itemize}
        \item Test pada dataset dari rumah sakit berbeda
        \item Validate across different demographics
        \item Ensure clinical applicability
    \end{itemize}
\end{enumerate}
\end{block}

\begin{block}{Prioritas MEDIUM}
\begin{enumerate}
    \setcounter{enumi}{2}
    \item \textbf{Add New Features}
    \begin{itemize}
        \item Konsultasi neurolog untuk biomarker tambahan
        \item Gait analysis, tremor measurements
        \item MRI/imaging data integration
    \end{itemize}
\end{enumerate}
\end{block}
\end{frame}

\begin{frame}{Rekomendasi Deployment}
\begin{block}{Immediate Actions (Ready Now)}
\begin{enumerate}
    \item ✓ \textbf{Deploy Current Model}
    \begin{itemize}
        \item 94.87\% sudah excellent untuk screening
        \item Gunakan web app atau CLI tool
        \item Implement confidence thresholds
    \end{itemize}
    
    \item ✓ \textbf{Integrate ke Workflow Klinis}
    \begin{itemize}
        \item Screening tahap 1: Model ML
        \item Screening tahap 2: Review dokter
        \item Diagnosis final: Neurolog spesialis
    \end{itemize}
    
    \item ✓ \textbf{Monitoring dan Logging}
    \begin{itemize}
        \item Log semua prediksi
        \item Track false positives/negatives
        \item Collect feedback untuk model update
    \end{itemize}
\end{enumerate}
\end{block}
\end{frame}

\begin{frame}{Future Research Directions}
\begin{block}{Pengembangan Jangka Panjang}
\begin{enumerate}
    \item \textbf{Multi-class Classification}
    \begin{itemize}
        \item Deteksi tingkat keparahan: Early, Moderate, Advanced
        \item Lebih actionable untuk treatment planning
    \end{itemize}
    
    \item \textbf{Deep Learning Approach}
    \begin{itemize}
        \item Neural networks untuk pattern yang lebih kompleks
        \item Memerlukan dataset yang jauh lebih besar
    \end{itemize}
    
    \item \textbf{Multi-modal Data}
    \begin{itemize}
        \item Combine voice + gait + MRI + blood tests
        \item Holistic assessment
    \end{itemize}
    
    \item \textbf{Longitudinal Analysis}
    \begin{itemize}
        \item Track progression over time
        \item Predict disease trajectory
    \end{itemize}
\end{enumerate}
\end{block}
\end{frame}

\begin{frame}{Impact dan Kontribusi}
\begin{block}{Kontribusi Ilmiah}
\begin{itemize}
    \item Comprehensive comparison 10 ML algorithms untuk Parkinson detection
    \item Systematic optimization dengan 5 strategies
    \item Evidence bahwa simple models cukup untuk small datasets
    \item Deployment-ready solution dengan multiple interfaces
\end{itemize}
\end{block}

\begin{block}{Kontribusi Praktis}
\begin{itemize}
    \item Screening tool yang cepat (<100ms) dan akurat (94.87\%)
    \item Accessible (web app + CLI) untuk berbagai user
    \item Cost-effective: Tidak perlu equipment mahal
    \item Scalable: Dapat digunakan untuk screening massal
\end{itemize}
\end{block}

\begin{block}{Clinical Value}
\begin{itemize}
    \item Early detection → Early intervention → Better outcomes
    \item Reduce burden pada neurolog dengan pre-screening
    \item Increase accessibility di daerah dengan limited specialists
\end{itemize}
\end{block}
\end{frame}

\begin{frame}{Kesimpulan Akhir}
\begin{center}
\Large
\textbf{Sistem Deteksi Parkinson dengan Machine Learning}

\vspace{0.5cm}
\normalsize
Berhasil mengembangkan model XGBoost dengan:

\vspace{0.3cm}
\begin{itemize}
    \item \textbf{Akurasi: 94.87\%}
    \item \textbf{ROC-AUC: 96.90\%}
    \item \textbf{Status: Production-Ready}
\end{itemize}

\vspace{0.5cm}
\textbf{Model sudah optimal} untuk dataset yang tersedia.

\vspace{0.3cm}
Improvement lebih lanjut memerlukan:
\begin{itemize}
    \item More data (500-1000+ samples)
    \item Better features (medical expert consultation)
    \item External validation
\end{itemize}

\vspace{0.5cm}
\Large
\textcolor{darkgreen}{\textbf{Siap untuk Deployment!}}
\end{center}
\end{frame}

\begin{frame}[standout]
\Huge
\textbf{Terima Kasih}

\vspace{1cm}
\Large
Questions?

\vspace{1cm}
\normalsize
\textbf{Presenter:}\\
Nafiz Ahmadin Harily\\
NIM: 122430051\\
Teknik Biomedis - ITERA

\vspace{0.5cm}
\textbf{Email:}\\
\texttt{nafiz.122430051@student.itera.ac.id}
\end{frame}

\begin{frame}{Appendix: Technical Specifications}
\begin{columns}
\column{0.5\textwidth}
\begin{block}{System Requirements}
\begin{itemize}
    \item Python 3.11.9
    \item RAM: 2GB minimum
    \item Storage: 100MB
    \item OS: Windows/Linux/Mac
\end{itemize}
\end{block}

\begin{block}{Key Libraries}
\begin{itemize}
    \item XGBoost 1.7.6
    \item Scikit-learn 1.3.0
    \item Pandas 2.0.3
    \item NumPy 1.24.3
\end{itemize}
\end{block}

\column{0.5\textwidth}
\begin{block}{Model Files}
\begin{itemize}
    \item xgboost.pkl (2.1 MB)
    \item scaler.pkl
    \item feature\_names.json
\end{itemize}
\end{block}

\begin{block}{Performance}
\begin{itemize}
    \item Inference: <100ms
    \item Throughput: 10+ pred/sec
    \item Memory: ~50MB RAM
\end{itemize}
\end{block}
\end{columns}
\end{frame}

\begin{frame}{Appendix: Dataset Features Details}
\begin{table}
\centering
\tiny
\begin{tabular}{lp{9cm}}
\toprule
\textbf{Feature} & \textbf{Description} \\
\midrule
MDVP:Fo(Hz) & Average vocal fundamental frequency \\
MDVP:Fhi(Hz) & Maximum vocal fundamental frequency \\
MDVP:Flo(Hz) & Minimum vocal fundamental frequency \\
MDVP:Jitter(\%) & Frequency variation \\
MDVP:Jitter(Abs) & Absolute jitter \\
MDVP:RAP & Relative amplitude perturbation \\
MDVP:PPQ & Pitch period perturbation quotient \\
Jitter:DDP & Average absolute difference of differences \\
MDVP:Shimmer & Amplitude variation \\
MDVP:Shimmer(dB) & Shimmer in decibels \\
Shimmer:APQ3 & Amplitude perturbation quotient (3-point) \\
Shimmer:APQ5 & Amplitude perturbation quotient (5-point) \\
MDVP:APQ & Amplitude perturbation quotient \\
Shimmer:DDA & Average absolute differences \\
NHR & Noise-to-harmonics ratio \\
HNR & Harmonics-to-noise ratio \\
RPDE & Recurrence period density entropy \\
DFA & Detrended fluctuation analysis \\
spread1, spread2 & Nonlinear measures of fundamental frequency \\
D2 & Correlation dimension \\
PPE & Pitch period entropy \\
\bottomrule
\end{tabular}
\end{table}
\end{frame}

\end{document}
